\subsection{Pressure}
The following finite difference equations were used to calculate the
pressure laplacian for the left boundary and interior points. 

\begin{itemize}
    \item Left boundary (i.e., $i=0$, coding index \texttt{p(:,0)})
        \begin{equation}
            p_{0,j} =   \frac{
                            p_{1,j} - 2p_{0,j} + p_{M,j}
                            }{\Delta x^{2}}
                            +
                        \frac{
                            p_{0,j+1} - 2p_{0,j} + p_{0,j}
                            }{\Delta y^{2}}
        \end{equation}
    \item Interior points (i.e., $i=1$ through $i=M$, coding index
        \texttt{p(:,1:M)})
        \begin{equation}
            p_{i,j} =   \frac{
                            p_{i+1,j} - 2p_{i,j} + p_{i-1,j}
                            }{\Delta x^{2}}
                            +
                        \frac{
                            p_{i,j+1} - 2p_{i,j} + p_{i,j}
                            }{\Delta y^{2}}
        \end{equation}
    \item Right boundary (i.e., $i=M+1$, coding index \texttt{p(:,M+1)}) \\
        \\
        Since the pressure is defined at the cell centers the pressure at 
        $i=M+1$ can be directly copied $i=0$, namely
        \texttt{p(:,M+1)=p(:,0)}.
\end{itemize}
            
