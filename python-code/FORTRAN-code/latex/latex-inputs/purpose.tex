\section{Purpose}
The purpose of this document is to serve as a ``running'' document to 
describe in detail the RANS/BHR solver used to solve a 1D flow problem 
including a description of the finite difference method, 
boundary conditions, and initial conditions. 

\section{Lid Cavity Problem}
The RANS/BHR code has been developed by making changes to a first order forward
in time central in space (FTCS) Navier-Stokes solver for a 2D lid cavity
problem solved on a staggered grid using a factorial step method.  
\subsection{Method}
Starting with the governing equations for the non-dimensional Navier-Stokes
equations,
\begin{subequations}
    \begin{equation}
        \div{\mathbf{u}} = 0
        \label{eq:masss}
    \end{equation}
    \begin{equation}
        \pdv{\mathbf{u}}{t} + \div \left({\mathbf{u}\mathbf{u}}\right) = 
            - \grad{p} + \frac{1}{Re} \laplacian{\mathbf{u}}
            \label{eq:momentum}
    \end{equation}
\end{subequations}                              
where Eqs.~\ref{eq:mass} and \ref{eq:momentum} represent the transport of mass
and momentum respectively. As stated above the solution to the lid cavity is
obtained using a factorial step method, which decouples the velocity and the
pressure terms by separating the momentum equation in to parts in order to
enforce the mass conservation restraint. This approach gives the following 
set of equations 
