\documentclass[12pt]{article}
\newcommand{\pdftitle}{1d-NS-equations}
\input{preamble}


\fancyhead[L]{1D Navier-Stokes Equations}
\fancyhead[C]{}
\fancyhead[R]{E. Torres}
\renewcommand{\headrulewidth}{1.0pt}

\fancyfoot[L]{\today}
\fancyfoot[C]{}
\fancyfoot[R]{\thepage}
\renewcommand{\footrulewidth}{1.0pt}

\begin{document}
\section{1D Navier-Stokes Equations}
\subsection{Derivation}
Starting with the 2D Navier-Stokes (NS) Equation, namely
\begin{subequations}
    \begin{equation}
        \pdv{u}{t} + \pdv{uu}{x} + \pdv{uv}{y} = 
                -\frac{1}{\rho} \pdv{p}{x} 
                + \nu \left( \pdv{u}{x}{x} + \pdv{u}{y}{y} \right) 
        \label{eq:NS-u}
    \end{equation}
    \begin{equation}
        \pdv{v}{t} + \pdv{uv}{x} + \pdv{vv}{y} = 
                -\frac{1}{\rho} \pdv{p}{y} 
                + \nu \left( \pdv{v}{x}{x} + \pdv{v}{y}{y} \right) 
        \label{eq:NS-v}
    \end{equation}
\end{subequations}
where for 1D all the gradients in the $y$-direction are zero 
($\pdv*{(\;)}{y}=0$), reducing Eqs.~(\ref{eq:NS-u},\ref{eq:NS-v}) to
\begin{subequations}
    \begin{equation}
        \pdv{u}{t} + \pdv{uu}{x} = 
            -\frac{1}{\rho} \pdv{p}{x}
            + \nu \pdv{u}{x}{x}
        \label{eq:NS-u-1d}
    \end{equation}
    \begin{equation}
        \pdv{v}{t} + \pdv{uv}{x} =
            \nu \pdv{v}{x}{x}
        \label{eq:NS-v-1d}
    \end{equation}
\end{subequations}
Next, we recognize that the transport of $u$ is independent of $v$ and can
be obtained solely from Eq.~(\ref{eq:NS-u-1d}).  Similarly to solving the
2D-NS equations, we can apply the fractional step method and separate
Eq.~(\ref{eq:NS-u-1d}) into the following steps 
\begin{subequations}
    \begin{equation}
        \pdv{u}{t} =  - \pdv{uu}{x} + \nu \pdv{u}{x}{x} 
        \hspace{0.25cm}
        : u^{n} \rightarrow u^{\ast}
        \label{eq:NS-un-1d}
    \end{equation}
    \begin{equation}
        \pdv{u}{t} = -\frac{1}{\rho} \pdv{p}{x}
        \hspace{0.25cm}
        : u^{\ast} \rightarrow u^{n+1}
        \label{eq:NS-ustar-1d}
    \end{equation}
\end{subequations}
Next, the pressure is calculated by taking the divergence of
Eq.~(\ref{eq:NS-ustar-1d}) and applying the continuity equation,  
\begin{subequations}
    \begin{align}
        -\frac{1}{\rho} \text{div}(\text{grad}(p)) & =
                \text{div}\bigg(\pdv{u}{t}\bigg)        \\
        -\frac{1}{\rho} \text{div}(\text{grad}(p)) & =
            \frac{
                    \cancel{\text{div}\big(u^{n+1}_{i+\frac{1}{2}}\big)} -
                    \text{div}\big(u^{\ast}_{i+\frac{1}{2}}\big)
                }
                {\Delta t}
    \end{align}
\end{subequations}
Thus
\begin{equation}
    -\frac{1}{\rho} \text{div}(\text{grad}(p)) =
        \frac{1}{\Delta t} \text{div} \big(u^{\ast}_{i+\frac{1}{2}}\big)
\end{equation}
\newpage
\subsection{Finite Difference Equations}
Discretetizing the above equations gives the following finite difference
formulas, starting with Eq.~(\ref{eq:NS-un-1d})

\colorboxed{
    \frac{u^{\ast}_{i+\frac{1}{2}} - u^{n}_{i+\frac{1}{2}}}{\Delta t} =
        -\frac{\big(u^{n}_{i+1}\big)^{2} - \big(u^{n}_{i-1}\big)^{2}}
        {\Delta x}
        +\nu 
        \frac{u^{n}_{i+\frac{3}{2}} - 2 u^{n}_{i+\frac{1}{2}} + u^{n}_{i-\frac{1}{2}}}
        {\Delta x ^{2}}
        }
Next, the finite difference scheme for the pressure calculation,
\colorboxed{
    \frac{1}{\rho} \left( \frac{p_{i+1} -2 p_{i} + p_{i-1}} {\Delta x^2} \right) =
            \frac{1}{\Delta t} \Bigg(\frac{u^{\ast}_{i+\frac{1}{2}} - u^{\ast}_{i-\frac{1}{2}}}
            {\Delta x}  \Bigg)
        }
Lastly the time advancing scheme,
\colorboxed{
    \frac{u^{n+1}_{i+\frac{1}{2}} - u^{\ast}_{i+\frac{1}{2}}}{\Delta t} =
        -\frac{1}{\rho}\left(\frac{p_{i+1} - p_{i}}{\Delta x} \right)
    }
        
\end{document}
